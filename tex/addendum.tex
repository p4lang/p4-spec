
\documentclass[12pt]{article}

\usepackage[bottom=3cm]{geometry}
\usepackage{p4doc}
\usepackage{fleqn}
\usepackage{fancyvrb}
\usepackage{graphicx}
\usepackage{tabularx}
\usepackage{soul}
\usepackage{float}
\usepackage{underscore}
\usepackage[table]{xcolor}

\usepackage[parfill]{parskip}

\usepackage{silence}
\WarningFilter{Fancyhdr}{\headheight is too small}
\WarningFilter{latex}{Overfull \hbox}


%%not_a_keyword auto
%%not_a_keyword default

%%not_a_keyword expression_local_variables

%%not_a_keyword ternary
%%not_a_keyword exact
%%not_a_keyword lpm

%%not_a_keyword length

%%not_a_keyword type

%%not_a_keyword reads
%%not_a_keyword actions
%%not_a_keyword min_size
%%not_a_keyword max_size
%%not_a_keyword size
%%not_a_keyword modifier
%%not_a_keyword support_timeout


\lstdefinestyle{BNFstyle}{
    language=BNF,%
    frame=single,%
    backgroundcolor=\color{bnfgreen},%
    morekeywords={}%
}

\lstdefinestyle{P4style}{
    language=C,%
    frame=single,%
    backgroundcolor=\color{codeblue},%
    keywords={},%
    basicstyle=\ttfamily,%
    aboveskip=3mm,%
    belowskip=3mm,%
    fontadjust=true,%
    keepspaces=true,%
    keywordstyle=\bfseries,%
    captionpos=b,%
    framerule=0.3pt,%
    firstnumber=0,%
    numbersep=1.5mm,%
    numberstyle=\tiny,%
}

\ifpdf
    \pdfinfo {
        /Author   (The P4 Language Consortium)
        /Title    (The P4 Language Specification, v 1.1.0 Addendum)
        /Keywords (P4)
        /Subject  ()
        /Creator  (TeX)
        /Producer (PDFLaTeX)
    }
\fi

\begin{document}
\vspace{2cm}

%\title{\sffamily\bfseries\huge The P4 Language Specification\\ \vspace{3mm} \Large Version 1.0.2}
\centerline{\sffamily\bfseries\huge The P4 Language Specification}
\vspace{3mm}
\centerline{\sffamily\Large Version 1.1.0 Addendum}
\vspace{3mm}
\centerline{\sffamily\large January 11, 2016}
\vspace{8mm}
\centerline{\sffamily\large The P4 Language Consortium}

\date{January 11, 2016}
\thispagestyle{firstpagestyle}
%\maketitle



\SECTION{Addendum: Preview of architecture-language separation}{separation}
The current P4 specification defines the language relative to an
abstract forwarding model with a specific architecture, as described
in Section~\ref{sec:absmod}.  The P4 Language Design group is working
towards the separation of language features from architecture features,
and this addendum gives a summary preview of ideas under consideration.

A machine that can run a P4 program is called \textit{target}.
While P4 provides a standard language for describing the logic within
programmable regions of a forwarding element, the programmable regions that are
actually available and the data flow between those regions will likely vary
from target to target.  For example, one target may consist of a parser,
ingress \matchaction pipeline and egress \matchaction pipeline, connected in
sequence.  Another target may consist of several parser-pipeline pairs,
which the packet may flow through in any order by setting appropriate
control signals.

With architecture-language separation, the P4 language would address
the contents of each programmable region, but would not attempt to standardize
any one architectural model. Instead, the overall P4 framework would provide
the facilities for target providers and standards bodies to define multiple
such architectures.  To accomplish this, each target would conform to a
\textit{Target Architecture}, specified partially as a collection of
P4 code, and partially as a set of non-P4 (or possible extended P4
in the future) specifications that describe the P4-programmable regions of
the target and how those regions interact with each other. Examples of
programmable regions would be packet parsers, and pipelines of tables
and actions.  The existing P4 abstract forwarding model would be one standard
example of a target architecture.



\SUBSECTION{Target Architectures}{archs}
A P4 \textit{Target Architecture} is the complete specification of a P4
target's programmable resources and the way those resources are connected.
The architecture consists of both P4 code and an accompanying rigorous
specification.


\SUBSUBSECTION{Target Architecture Structure}{archsstruct}
The P4 portion of a target architecture description provides
\textit{prototypes} for the programmable regions of the target. These
prototypes specify the input and output interfaces to each region, including
the format of metadata passed across each interface.  These interfaces form
the conection between the region of P4 code in question and the surrounding
non-P4-programmable environment.  For instance, a region of code may receive
intrinsic metadata reporting a packet's ingress port and length, and may write
intrinsic metadata controlling the packet's egress port and queue priority.

The P4 portion of the target architecture description also provides
definitions of the available primitive actions (Section~\ref{sec:actions}) and
extern object types (Section~\ref{sec:externs}), which represent the
fundamental processing capabilities of the target.  Examples of these include
arithmetic functions and checksum generators.

Together with these two P4-described portions, the additional non-P4
specification clarifies the meaning of the definitions in the P4 portion of
the architecture and explains how they fit together. Initially, this is
envisaged to be mostly human-language documentation and visual diagrams to
show the flow of data between programmable regions, though it may also
contain pseudocode to rigorously specify the behavior of logic not expressible
in P4.  Looking further out, there has already been experimentation with
some extension of P4 itself, to allow rigorous specification of the interaction
between regions in terms of established P4 mindset and terminology.


\SUBSUBSECTION{Target Architecture Selection}{archsselect}
A program's target architecture is selected by including that architecture's P4
library in the source code, and then writing structures that conform to the
prototypes it specifies and using the actions and externs that it provides.

No one architecture is mandated by the P4 spec, and a given physical target may 
support multiple architectures.  Some architectures may be written by a target
provider and highly specialized to the underlying machinary, while others may
be standardized and intentionally abstract to allow greater portability and
ease-of-use.  A particular example of the latter is that a standard switch
architecture will be defined based on the existing abstract forwarding model.

An important aspect is that all P4 programs written for a given architecture
are portable across all targets that faithfully implement that architecture
(assuming that enough resources are available).  P4 conformance of a target is
defined as follows: if a specific target supports a given target architecture,
then a program written to that architecture and executed on the target must
provide exactly the same behavior as the same program executed on an abstract
machine with infinite resources.

In general, P4 programs are not expected to be portable across different
architectures.  For example, executing a P4 program that controls packet
broadcast by writing special intrinsic metadata will not work on a target that
provides no such intrinsic metadata.  Further, particular targets may not
support fully some P4 language constructs (for example, some targets may not
support features necessary for IPv4 options processing or arbitrary-length
stacked protocol headers).  Ideally any restrictions on the P4 language
imposed by a specific target should be clearly documented by the target
architecture.  At the very least, restrictions have to be conveyed to P4
programmers using clear compiler error messages when attempting to compile
programs that use unsupported features.



\SUBSECTION{Programmable blocks}{blocks}
Programmable blocks are user-defined blocks of P4 code that can be
instantiated multiple times within a program, and interact with the enclosing
target architecture by occupying its programmable regions.  Each instance
of a programmable block matches a P4-described prototype in the
architecture specification.


\SUBSUBSECTION{Programmable block types}{pbtypes}
A programmable block type is comprised of a signature and code body. The body
forms a new scope that can contain any normal P4 declaration. The enclosed
code is lexically scoped and additionally has access to the external
metadata parameters declared by its input-output signature.

Similarly to header types for example, the objects declared inside a
programmable block type do not actually "exist" inside the program until
the block is \textit{instantiated}.  In this sense, a programmable block
type is declaring a "template" of P4 code that can be stamped down into the
program.


\SUBSUBSECTION{Programmable block instances}{pbinstances}
An instance of a programmable block type represents concrete resource
declarations of the contents of the block.  Because of this, blocks cannot
be instantiated dynamically at run time: they are static, compile-time
declarations.

When creating an instance, the programmer must bind all of the input-output
parameters in the type's signature either to constants or other object names
that are currently in scope. 

Multiple instances of the same block type create completely separate
instances of the type's component objects which the surrounding architecture
and/or a runtime API can refer to using dotted notation.


\SUBSUBSECTION{Programmable block prototypes}{pbprototypes}
Target architectures use programmable blocks to segment P4 code into the
various programmable regions of the underlying target.  The architecture
specifies the prototypes of the blocks it expects to be filled in by the
program.  These prototypes specify the signature of a block but leave its
implementation undefined.  They are expected to be paired with a concrete
programmable block declaration that has a matching signature.

Prototypes may also include type variables, which are resolved to
concrete types when the prototype is paired with its implementation.  The
identifiers in a prototype's type variable list are available as valid types
for the parameters in the prototype's signature. These type variables provide
a mechanism for architectures to pass user-defined structs of header instances
between P4 code blocks without mandating ahead of time what those structs are.

A target architecture may specify several prototypes for identical underlying
resources (such as $n$ prototypes for $n$ separately programmable yet
functionally identical hardware parsers). If a program uses the same
implementation for each of these resources, it can use \textit{typedef} to
alias the shared implementation's programmable block type to all of the
prototypes expected by the architecture.

While not explicitly disallowed, P4 programmers are unlikely to find much
benefit from writing their own prototypes.  Their utility is in target
architecture specification only.



\SUBSECTION{Standard Library}{stdlib}
The P4 portion of a target architecture description provides
definitions of the available primitive actions and extern object types.
To promote portability of P4 programs, a standard library of actions and externs
for common packet processing operations is defined.  While targets may provide
target-specific definitions that offer more specific and finely-tuned
functionality, this library provides more generalized functionality that
all targets should be able to support.

In addition, the definition of a standard library of extern object types
assists in simplifying the P4 language, since the function of many constructs
currently in the language can be delegated to extern objects, thus simplifying
the core P4 language significantly.


\SUBSUBSECTION{Primitive Actions}{stdlib-primitives}
The primitive actions are standard and expected to be supported
by \textit{all} targets, regardless of the target architecture being used.
The list of library actions is a subset of the current P4 list, as
given in Section~\ref{sec:primitiveactions}:

\begin{table}[H]
\begin{center}
\begin{tabular}{| l | p{.5\textwidth} |} \hline
\textbf{Name} &
\textbf{Summary} \\ \hline
\texttt{add_header} &
Add a header to the packet's Parsed Representation \\ \hline
\texttt{copy_header} &
Copy one header instance to another. \\ \hline
\texttt{remove_header} &
Mark a header instance as invalid. \\ \hline
\texttt{modify_field} &
Set the value of a field in the packet's Parsed Representation. \\ \hline
\texttt{no_op} &
Placeholder action with no effect. \\ \hline
\texttt{push} &
Push all header instances in an array down and add a new header at the top. \\ \hline
\texttt{pop} &
Pop header instances from the top of an array, moving all subsequent array elements up. \\ \hline
\end{tabular}
\end{center}
\caption{Standard Primitive Actions}
\label{tab:primitiveactions}
\end{table}


\SUBSUBSECTION{Parser Exceptions}{standardparserex}
The parser exceptions are standard, regardless of target architecture.
The prefix "pe" stands for parser exception.  The list of parser
exceptions is as in Section~\ref{sec:parserexceptions}, with one addition:

\begin{table}[H]
\begin{center}
\begin{tabular}{| l | p{.6\textwidth} |} \hline
\textbf{Identifier} &
\textbf{Exception Event} \\ \hline
\texttt{p4_pe_index_out_of_bounds} &
A header stack array index exceeded the declared bound. \\ \hline
\texttt{p4_pe_out_of_packet} &
There were not enough bytes in the packet to complete an extraction operation. \\ \hline
\texttt{p4_pe_header_too_long} &
A calculated header length exceeded the declared maximum value. \\ \hline
\texttt{p4_pe_header_too_short} &
A calculated header length was less than the minimum length of the fixed length 
portion of the header. \\ \hline
\texttt{p4_pe_unhandled_select} &
A select statement had no default specified but the expression value was not 
in the case list. \\ \hline
\texttt{p4_pe_data_overwritten} &
A given header instance was extracted multiple times. \\ \hline
\texttt{p4_pe_checksum} &
A checksum error was detected. \\ \hline
\texttt{p4_pe_default} &
This is not an exception itself, but allows the programmer to define a handler 
to specify the default behavior if no handler for the condition exists. \\
\hline
\end{tabular}
\end{center}
\caption{Standard Parser Exceptions}
\label{tab:parserexceptions}
\end{table}


\SUBSUBSECTION{Stateful Objects}{stdlib-state}
Counters, meters and registers maintain state for longer than one packet. 
Together they are called stateful memories.  These are described in
Section~\ref{sec:stateful}.  They are accessed via respective extern
object types in the standard library.  Generic method calls on these objects
replace the earlier custom P4 syntax.


\SUBSUBSECTION{Checksums and Calculations}{stdlib-calc}
Checksums and hash value generators are examples of functions that operate on a
stream of bytes from a packet to produce an integer.  These are described
in Section~\ref{sec:checksandhash}.   They are accessed via respective extern
object types in the standard library.  Generic method calls on these objects
replace the earlier custom P4 syntax.


\SUBSUBSECTION{Action profiles}{stdlib-actprof}
In some instances, action parameter values are not specific to a match entry but
could be shared between different entries. Some tables might even want to share
the same set of action parameter values. This can be expressed in P4 with
action profiles.  These are described in Section~\ref{sec:actprofdecs}.  They
are accessed via an extern object type in the standard library.
Generic method calls on these objects replace the earlier custom P4 syntax.
Action profiles are an example of a table modifier extern object type.


\SUBSUBSECTION{Digests}{stdlib-digest}
Digests serve as a generic mechanism to send data from the middle of a P4 block
to an external non-P4 receiver. This receiver can be anything from a
fixed-function piece of hardware to a control-plane function.  The
\textit{generate_digest} primitive action is described in
Section~\ref{sec:primitiveactions}.  This is accessed via an extern object
type in the standard library.  A generic method call on such objects replaces
the earlier custom P4 action.



\SUBSECTION{Standard Switch Architecture}{simplearch}
The Standard Switch Architecture defines a highly abstract packet forwarding
architecture geared towards packet switching.  It serves as:
\begin{itemize}
\item
An example P4 target architecture specification; and
\item
A widely supported architecture for simple yet portable P4 programs
\end{itemize}
While this architecture is designed primarily to allow the expression of packet
switching programs, it is flexible enough to implement more advanced behavior.
Other simple architectures geared towards different environments, such as NICs,
could also be defined.  The architecture is described in
Section~\ref{sec:absmod}.


\SUBSUBSECTION{Programmable regions}{simplearch-prototypes}
The Simple Switch Architecture has three P4-programmable regions: parser,
ingress, and egress.  It provides prototypes for these.  Note that this gives
a more explicit meaning to the blocks declared in traditional P4 programs.
A draft form of the intrinsic metadata associated with the interfaces
to these regions is given next.


\SUBSUBSECTION{Intrinsic Metadata}{simplearch-intrinsicmd}
All three blocks receive a read-only metadata header containing basic
information about the packet:

%%code
\begin{lstlisting}[style=P4style]
header_type packet_metadata_t {
    fields {
        bit<16> ingress_port; // The port on which the packet arrived.
        bit<16> length;       // The number of bytes in the packet. 
                              // For Ethernet, does not include the CRC. 
                              // Cannot be used if the switch is in
                              // 'cut-through' mode.
        bit<8>  type;         // Represents the type of instance of
                              // the packet: 
                              //   - PACKET_TYPE_NORMAL
                              //   - PACKET_TYPE_INGRESS_CLONE
                              //   - PACKET_TYPE_EGRESS_CLONE
                              //   - PACKET_TYPE_RECIRCULATED
                              // Specific compilers will provide macros
                              // to give the above identifiers the
                              // appropriate values
    }
}
\end{lstlisting}
%%endcode

The ingress block also receives the exit result of the parser: 

%%code
\begin{lstlisting}[style=P4style]
header_type parser_status_t {
    fields {
        bit<16> return_code;       // The final status of the parser.
                                   // 0 if parser returned 'accept'
                                   // TODO: Define other values 
        bit<8>  user_error_data;   // An opaque value written by
                                   // user-defined parser exceptions
    }
}
\end{lstlisting}
%%endcode

The ingress block's output intrinsic metadata controls how the packet will be
forwarded, and possibly replicated:

%%code
\begin{lstlisting}[style=P4style]
header_type ingress_pipe_controls_t {
    fields {
        bit<16> egress_spec;   // Specification of an egress.
                               // This is the 'intended' egress as
                               // opposed to the committed physical
                               // port(s).
                               //
                               // May be a physical port, a logical
                               // interface (such as a tunnel, a LAG,
                               // a route, or a VLAN flood group) or
                               // a multicast group.
        bit     drop;          // Do not send the packet on to the
                               // queueing system. Other functions
                               // like copy-to-cpu and clone will
                               // still occur.
        bit     copy_to_cpu;   // Send a copy of the packet to the
                               // slow path.
        bit<8>  cpu_code;      // Opaque identifier packaged with
                               // the packet, when sending to the
                               // slow path.
    }
}
\end{lstlisting}
%%endcode

The egress block receives further read-only information about the packet
determined while it was in the queueing system:

%%code
\begin{lstlisting}[style=P4style]
header_type egress_aux_packet_metadata_t {
    fields {
        bit<16> egress_port;      // The physical port to which this
                                  // packet instance is committed.
        bit<16> egress_instance;  // An opaque identifier differentiating
                                  // instances of a replicated packet.
    }
}
\end{lstlisting}
%%endcode

The egress block's output intrinsic metadata no longer has access to the egress
spec for writing, since the packet has already been committed to a physical
port:

%%code
\begin{lstlisting}[style=P4style]
header_type egress_pipe_controls_t {
    fields {
        bit     drop;          // Do not send the packet out of its
                               // egress port. Other functions
                               // like copy-to-cpu and clone will
                               // still occur.
        bit     copy_to_cpu;   // Send a copy of the packet to the
                               // slow path.
        bit<8>  cpu_code;      // Opaque identifier packaged with
                               // the packet, when sending to the
                               // slow path.
        bit     recirculate    // If true, recirculate packet to
                               // ingress parser
    }
}
\end{lstlisting}
%%endcode


\SUBSUBSECTION{Egress Port Selection, Replication and Queuing}{simplearch-egress}
The Simple Switch Architecture's egress mechanism is as described
in Section~\ref{sec:egress}.  This is a mechanism that is provided by
this particular architecture, rather than something inherent to P4.


\SUBSUBSECTION{Cloning, Mirroring, Resubmission and Recirculation}{simplearch-clone}
The Simple Switch Architecture's cloning, mirroring, and resubmission
and recirculation mechanism are as described in Section~\ref{sec:recirc}.
These involve primitive actions that are provided by this particular
architecture, rather than actions inherent to P4.



\end{document}
